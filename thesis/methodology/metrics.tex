\section{Performance Metrics}
\label{sec:methodology:metrics}

When we simulate our workloads we expect destructive interefernce between benchmarks to cause slowdowns.
Performance metrics are need to quantify the performance of workloads, and in order to compare performance of different cache partitioning algorithms.
We define two metrics; system throughtput (STP) and harmonic mean of speedup (HMS).

In order to define STP and HMS, two concepts are needed; private mode execution time and shared mode execution time.
Shared mode execution time is the time it takes for a benchmark to complete when run as a part of a workload.
Private mode execution time is defined as the time it takes for a benchmark to complate when run alone on the same processor model.
By definition we expect a benchmark to execute slower in shared mode than in private mode.

Based on private and shared mode execution time we can define normalized progress (NP) as $NP_i = \frac{T^{P}_i}{T^{S}_i}$.
Here $T^{P}_i$ and $T^{S}_i$ is respectively the private and shared mode execution time for benchmark $i$.
Normalized progress is a measure of benchmark progression in shared mode.
A perfect value of 1 indicates that the benchmark progresses just as fast in shared and private mode.
While a value of 0.5 indicates that the benchmark progresses at half the rate in shared mode compared to private mode.
STP as defined by L. Eeckhout~\cite{Eeckhout2010} as the sum of NP for all benchmarks in a worklaod, as shown in equation~\ref{eq:STP}
By definition a perfect STP value equals the count of benchmark in a workload, in our case either 4, 8 or 16.

\begin{equation} \label{eq:STP} 
 STP = {\sum\limits_{n=1}^{k}}\frac{T^{P}_i}{T^{S}_i}
\end{equation}

Also based on the definition of $NP_i$, L. Eeckhout~\cite{Eeckhout2010} defines HMS or hmean as shown in equation~\ref{eq:HMS}.
HMS is by definition the reciprocal of average normalized turnaround-time (ANTT), and hence HMS has a system level meaning relating to the average benchmark's normalized turnaround-time.

\begin{equation} \label{eq:HMS}
 HMS = \frac{k}{\sum\limits_{n=1}^{k}}\frac{Cycles^{P}_i}{Cycles^{S}_i}
\end{equation}

In general an increase in system throughput is expected to also improve turnaround time.
However, there are cases where this is not true.
A simple example would be a scheduler that prioritizes small jobs, hence improving throughput and STP, but by leaving long jobs in the queue turnaround time will increase and hence HMS will decrease.
It is therefore improtant to use both metrics to gain good insight into system performance.

In addition to HMS and STP, we will use misses per kilo instruction (mpki) when evaluating cache paritioning algorithms.
mpki is simply defined as the total number of misses in the LLC, per 1000 instructions.
mpki an important metric in our experiments, especially because we are simulating an out of order core. 
Due to latency hiding in the processor we cannot assume that a reduction in mpki necessarily will increase performance nor that a performance increase must be caused by a reduction in mpki.
As a result, all three metrics are required to gain full insight into the effects of various cache partitioning algorithms.