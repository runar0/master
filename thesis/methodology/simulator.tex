\section{Simulator}
\label{sec:methodology:simulator}

There are several different simulators used in computer architecture research today~\cite{Carlson2011a, Binkert2011, Miller2010, Pellauer2011}.
Of the simulators used today, we evaluated two possible candidates for this work, Sniper~\cite{Carlson2011a} and gem5~\cite{Binkert2011}.
These two candidates are mainly chosen because they are both in active use by the CARD research group at NTNU where this thesis is performed.
Gem5 is also an obvious candidate because it is the simulator used in most computer architecture research today~\cite{Chen-Han2014}.
It is a cycle accurate simulator meaning that it theoretically can perfectly mimic real hardware.
This accuracy comes at the cost of simulation time.
The other candidate, Sniper, is based on interval simuation~\cite{Genbrugge2010}. 
Interval simulation allows it to simulate benchmarks significantly faster than gem5~\cite{Carlson2011a, Olsen2014} at the cost of reduced accuracy.
In addition, Sniper is multithreaded, and each simulated core runs in a separate thread.
This separation means Sniper can take advantage of today's multicore computers during simulations, gem5, on the other hand, performs all simulation in a single thread.
By having multiple threads there is a chance of clock skewing~\cite{Carlson2011a} during simulations in Sniper.
Clock skewing is when one core simulates faster or slower than the others, making the clock values in each core different.
When this happens, the simulator cannot correctly simulate inter-core interactions, such as access to the shared cache.
There are however techniques implemented in Sniper that attempts to reduce errors caused by clock skewing.

Based on the authors previous work~\cite{Olsen2014}, we chose Sniper over gem5 because the simplicity and speed of Sniper outweights the reduced accuracy compared to gem5.
Because we know that clock skewing in Sniper might be an issue for our work, we will perform an experiment investingating this error source further.