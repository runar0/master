\subsection{DRRIP}
\label{sec:background:algorithms:drrip}

\begin{figure}[ht]
    \centering
    \begin{tabular}{|p{2cm}|p{2cm}|p{2cm}|p{2cm}|}
        \hline
        A & B & C & D \\
        \hline
    \end{tabular}
    \begin{tabular}{p{2cm}p{2cm}p{2cm}p{2cm}p{2cm}p{2cm}}
        & 2 & 1 & 1 & 0 & \bf{Hit C} \\
        &   &   &   &   &
    \end{tabular}    

    \begin{tabular}{|p{2cm}|p{2cm}|p{2cm}|p{2cm}|}
        \hline
        A & B & C & D \\
        \hline
    \end{tabular}
    \begin{tabular}{p{2cm}p{2cm}p{2cm}p{2cm}p{2cm}p{2cm}}
        & 2 & 1 & 0 & 0 & \bf{Miss E} \\
        &   &   &   &   &
    \end{tabular}     

    \begin{tabular}{|p{2cm}|p{2cm}|p{2cm}|p{2cm}|}
        \hline
        E & C & A & B \\
        \hline
    \end{tabular}
    \begin{tabular}{p{2cm}p{2cm}p{2cm}p{2cm}p{2cm}p{2cm}}
        & 2 & 2 & 1 & 1 & \bf{Hit E} \\
        &   &   &   &   &
    \end{tabular}    

    \begin{tabular}{|p{2cm}|p{2cm}|p{2cm}|p{2cm}|}
        \hline
        E & C & A & B \\
        \hline
    \end{tabular}
    \begin{tabular}{p{2cm}p{2cm}p{2cm}p{2cm}p{2cm}p{2cm}}
        & 1 & 2 & 1 & 1 & 
    \end{tabular} 

    \caption{DRRIP managed 4-way cache set (M=2, static insertion)}
    \label{fig:background:drrip_example}
\end{figure}

Dynamic Re-Reference Interval Prediction (DRRIP) was first proposed by A. Jaleel et al.~\cite{Jaleel2010} in 2010.
DRRIP uses the concept of a re-reference chain (RRC).
This chain represents the order in which the algorithm expects cache blocks in the cache set to be re-referenced.
This chain is theoretically no different from the LRU stack envisioned in both LRU and TADIP. 
Both LRU and DRRIP has the same replacement policy, the element at the end of the re-reference chain or in the bottom of the LRU stack is replaced when a new block is inserted.
The assumption in both algorithms is that blocks higher up in the chain or the stack, has a higher probability of a re-reference.

While the elements in the LRU stack are strictly ordered, from the MRU to the LRU position, elements in the RRC are more loosely ordered.
All blocks have a re-reference interval between 0 and $2^M - 1$ with M being a configurable parameter.
There may be multiple blocks with the same re-reference interval, and the end of the chain is the block with the lowest re-reference interval.
If multiple blocks have the same value the original authors selects the first block found when scanning the cache blocks from left to right as the smallest.
On a miss, the algorithm will scan the cache set looking for a block with a distant re-reference interval ($2^M - 1$). 
If no block has this value the re-reference interval of all blocks is incremented by one. 
This process is repeated until a victim block is selected.

Unlike LRU, a cache block is not promoted to a near-immediate position on access under DRRIP. 
The DRRIP promotion policy is to promote a block by one position each reference unless it already is at a near re-reference interval (0).
The assumption here is that frequently used data is often reused, unlike LRU that assumes that recently used data is often reused.

Like TADIP, DRRIP has two different insertion policies Static RRIP (SRRIP) and Binominal RRIP (BRRIP).
SRRIP inserts new blocks with a long re-reference interval $2^M - 2$.
BRRIP, on the other hand, is analog to BIP in TADIP. 
It will in most cases insert new blocks with a distant re-reference interval $2^M - 1$ and only sometimes insert at the long re-reference interval as SRRIP.
The goal of BRRIP is the same as for BIP.
When mostly inserting new blocks near the end, trashing workloads will have a limited effect on the overall cache set.
This policy allows parts of the working set to stay in the cache while most new blocks get evicted early.

Selecting between the two insertion policies can be done in multiple ways, like covered in the description of TADIP.
The authors opted for using set-dueling in their original paper, and we opt to do this in our implementation as well.
Since there is lacking information regarding selection of duel sets in the original paper, we utilize the hashing function that we presented for TADIP in our implementation.

Figure~\ref{fig:background:drrip_example} shows an example cache managed by DRRIP.
In the example, M is set to 2, making the distant re-reference interval 3 and the long re-reference interval 2. 
In addition, we assume static insertion.
Initially, there are four blocks A, B, C and D with respectively re-reference intervals 2, 1, 1 and 0.
First an access hits the C block, and its value is decremented to 0.
Next a miss to block E occurs, and as no block has a re-reference interval of 3 the value of all blocks is incremented by one. 
A then has a value of 3 and is evicted, E in inserted in its place with a value of 2.
Finally, an access to E occurs causing its value to decrease.

\todo{Cover trash and stream resistance, there are some formulas in the original paper we might want to include}

\todo{We need to run an experiment evaluating how many duel-sets to use, the value of M and the binomial parameter}