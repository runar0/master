
\chapter{Discussion}
\label{cpt:discussion}


\section{Lack of implementation details}

\todo{We should mention the lack of implementation details available for some of the implemented algorithms. We had to make some assumptions along the way that will have an effect on our results compared to those of the original authors.}

\section{Parameter Fitting}

\todo{We have used the parameters given in their original papers, the parameters a possibly fitted towards the original authors workloads. We could most likely have acheived better speedups if we had fitted the parameters to our workloads. But then again is that better, given that we cannot fit to all possible workloads?}


\section{Clock Skew Synchronization Barrier}

One of the advantages of Sniper is the use of multiple simulation threads~\cite{Carlson2011a}.
We covered how this could improve performance in section~\ref{sec:methodology:simulator}, but this speed improvement does naturally decrease accuracy.
In the context of this thesis, there are two effects of having multiple
simulation threads that could bias our results.
The problem relates to the ordering of memory events.
With multiple simulation threads, the only guarantee the simulator gives regarding the ordering of memory events from different cores is that events from different intervals execute in the correct order.
The order of events from different cores within the same interval depends on the os scheduler.
The worst case would be if the os schedule all simulation threads serially and hence memory events would be sorted by time but grouped by origin core.
All the implemented cache partition algorithms assume that memory requests are in order, and this inaccuracy will break this assumption.
In addition to the effects on the cache partitioning, memory bus scheduling is hard to estimate when requests come out of order, this was an issue covered in the authors autumn project~\cite{Olsen2014}.

Section~\ref{sec:results:csmb_sensitivity} presents an experiment where we attempted to vary the CSMB interval, and we evaluate the performance using changes in measure STP and HMS.
The results of this experiment showed that there is little change when lowering the CSMB from the default value of 100 cycles all the way to 1 cycle.
This would indicate that although the potential issues outlined above are serious, they seem to not effect results when we synchronize every 100 cycles.
The cache used in this experiment and all other four core experiments is 4MB with 32 blocks per set.
In other words, there are 2048 sets.
For out of order memory accesses to have an impact on our accuracy there would have to be at least one of the 2048 sets accessed by two cores within the same interval.
Additionally the order of the requests have to be reversed.
For this to have a serious effect, we expect there would have to be a significant larger number of requests resolving the same set within the same interval.
This seems to be an unlikely situation, and our experiment supports this assumption.

The experiment only did a sensitivity analysis of 4-core workloads. 
We chose to do this because repeating it for 8- and 16-cores would have been unpractical given the time constraints of this work. 
However, as we scale the number of sets in the cache linearly with the number of cores, there is no reason to doubt that the results would have been the same in these cases.