
\section{Architectural overview}

\todo{Insert figure here}

\begin{table}[ht]
\centering
\begin{tabular}{rl}
\toprule
\bf{Processor core}                 & 3GHz 8 inst. dispatch width,          \\
                                    & 128 rob entries, 8 inst. commit width \\
\bf{Private L1 inst. and data cache}& 32kB, 64B line-size, 16-way, 8 mshrs, LRU \\
\bf{Private L2 unified cache}       & 128/256/512/1024kB, 64B line-size, 16-way, \\
                                    & 12 mshrs, LRU      \\
\bf{Shared L3 cache}                & 4/8/16/32MB, 64B line-size, 24 mshrs, \\
                                    & 32-way, varying replacement algorithm         \\
\bf{Memory controller}              & 12.8GB/s, 45ns access latency         \\
\bottomrule                             
\end{tabular}
\caption{Model properties}
\label{tbl:processor_model:properties}
\end{table}

Throughout this thesis, we utilize a CMP model simulated on Sniper~\cite{Carlson2011a} to obtain our results. 
In our model, each processing core has two levels of private cache, the L1 data and code caches and a unified L2 cache.
Additionally there is a third cache level, L3, which is shared by all cores. 
The private caches are managed by an LRU replacement policy, and the replacement policy of the third cache level varies throughout our experiments.
Figure~\todo{ref} shows an overview of the simulated architecture and Table~\ref{tbl:processor_model:properties} contains an overview of the system properties.
We base our first level cache sizes on current CMP architectures like the Intel Haswell architecture~\cite{Jain2013}. 
In the two lower levels, we use varying cache sizes in order to investigate the effect this has on the partitioning algorithms.
All cache timings are derived using CACTI 6.5\todo{ref} assuming a core frequency of 3GHz.

\begin{table}[ht]
\centering
\begin{tabular}{rl}
\toprule
\bf{Size}               & 32kB              \\
\bf{Block size}         & 64B               \\
\bf{Associativity}      & 16                \\
\bf{Banks}              & 1                 \\
\bf{Technology}         & 32nm              \\
\bf{Access time (Tag)}  & 0.16ns \\
\bf{Access time (Data)} & 0.62ns \\
\bf{Access cycles (Tag)}  & 1 \\
\bf{Access cycles (Data)} & 2 \\
\bottomrule
\end{tabular}
\caption{L1 cache properties}
\label{tbl:processor_model:l1}
\end{table}

Both the first level caches have a size of 32kB; divided into sets of 16 lines where each line is 64-bytes long.
Table~\ref{tbl:processor_model:l1} summaries these values as well as the optimal bank count\footnote{We define optimal bank count as the number of banks that provide the lowest access delay for data and tags measured in cycles.} and access times for the tag directory and data as estimated by CACTI. 
The access times are converted to cycles assuming a period of 0.3ns or a core clock of 3GHz. 
A tag access time of 0.16ns equals one cycles while a data access time of 0.62ns equals two cycles.

\begin{table}[ht]
\centering
\begin{tabular}{rrrrr}
\toprule
\bf{Size}                 & 128kB       & 256kB       & 512kB       & 1024kB            \\
\bf{Block size}           & 64B         & 64B         & 64B         & 64B               \\
\bf{Associativity}        & 16          & 16          & 16          & 16                \\
\bf{Banks}                & 1           & 2           & 4           & 4                 \\
\bf{Technology}           & 32nm        & 32nm        & 32nm        & 32nm              \\
\bf{Access time (Tag)}    & 0.28ns      & 0.29ns      & 0.33ns      & 0.39ns            \\
\bf{Access time (Data)}   & 0.66ns      & 0.83ns      & 1.18ns      & 1.08ns            \\
\bf{Access cycles (Tag)}  & 1           & 1           & 2           & 2                 \\
\bf{Access cycles (Data)} & 3           & 3           & 4           & 4                 \\
\bottomrule
\end{tabular}
\caption{L2 cache properties}
\label{tbl:processor_model:l2}
\end{table}

For the unified second level cache we have modelled four different sizes; 128kB, 256kB, 512kB and 1024kB. 
In all configurations there are 16 lines in each set and each line is 64-bytes long.
Using CACTI, we have found the optimal number of banks per size configuration and the corresponding access delays.
Table~\ref{tbl:processor_model:l2} summarises these values and shows the CACTI estimated access times. 
Again we have converted access times to cycle assuming a 0.3ns period.

\begin{table}[ht]
\centering
\begin{tabular}{rrrrr}
\toprule
\bf{Size}                 & 4MB         & 8MB         & 16MB        & 32MB              \\
\bf{Block size}           & 64B         & 64B         & 64B         & 64B               \\
\bf{Associativity}        & 32          & 32          & 32          & 32                \\
\bf{Banks}                & 4           & 4           & 4           & 8                 \\
\bf{Technology}           & 32nm        & 32nm        & 32nm        & 32nm              \\
\bf{Access time (Tag)}    & 0.52ns      & 0.71ns      & 0.88ns      & 0.85ns            \\
\bf{Access time (Data)}   & 1.81ns      & 2.11ns      & 2.69ns      & 3.30ns            \\
\bf{Access cycles (Tag)}  & 2           & 3           & 3           & 3                 \\
\bf{Access cycles (Data)} & 6           & 7           & 9           & 10                \\
\bottomrule
\end{tabular}
\caption{L3 cache properties}
\label{tbl:processor_model:l3}
\end{table}

Like the previous level we have four different size configurations for the third cache level; 4MB, 8MB, 16MB and 32MB.
Unlike the previous two cache levels, that all use a standard LRU replacement policy, we will vary the replacement policy of the third level.
The algorithms we are experimenting with in this work all resemble some form of way-partitioning by assigning a number of ways (or cache lines) per cache set to each core.
For this reason, the third level cache has 32 cache lines per cache set, giving us an average of 8/4/2 sets per core during our 4/8/16 core experiments.
The line size is set to be 64-bytes as in the previous level.
Again using CACTI we find the optimal bank count and the corresponding access times.
Table~\ref{tbl:processor_model:l3} summarises the cache properties for the various cache sizes.



\todo{Explain core interconnection network}


