\section{Clock Skew Sensitivity}
\label{sec:results:csmb_sensitivity}

As explained in Section~\ref{sec:framework:simulator} one of the techniques that make Sniper faster than conventional cycle-accurate simulators, such as gem5, is the use of multiple simulation threads that each simulate one processor core.
A method that keeps the simulation threads in sync is required to simulate inter-core interactions correctly.
The method used to keep the threads in sync affects both simulation accuracy and simulation time.
By having a relaxed synchronization method, one can improve simulation time, at the cost of simulation accuracy.
In our experiments, we have used barrier synchronization with a barrier width of 100 cycles.
Any inter-core interactions that occur between two successive barriers are not guaranteed to occur in the correct order, but events separated by a barrier will be simulated in the correct order.
In other words, within a single barrier there is the possibility that all simulation threads run sequentially.
For our work, this implies that there is a possibility that memory requests within a barrier is sent to the cache sorted by the core id, rather than by time.
Because of this possibility we expect that changing the clock skew minimization barrier (CSMB) value could have a noticeable effect on our experimentation results.

\begin{figure}[th]
    \centering
    \begin{subfigure}[b]{0.5\textwidth}
        \includegraphics[width=0.8\textwidth]{figures/results/speedup/csmb-stp-0128k-0100-csmb-4}
        \caption{STP sensitivity to CSMB.}
        \label{fig:results:csmb:stp}
    \end{subfigure}%
    \begin{subfigure}[b]{0.5\textwidth}
        \includegraphics[width=.8\textwidth]{figures/results/speedup/csmb-hms-0128k-0100-csmb-4}
        \caption{HMS sensitivty to CSMB.}
        \label{fig:results:csmb:hms}
    \end{subfigure}
    \begin{subfigure}[b]{0.6\textwidth}
        \includegraphics[width=.8\textwidth]{figures/results/speedup/csmb-walltime-0128k-0100-csmb-4}
        \caption{walltime sensitivity to CSMB.}
        \label{fig:results:csmb:walltime}
    \end{subfigure}
    \caption{STP, HMS and walltime sensitivity to size of CSMB.}
    \label{fig:results:csmb}
\end{figure}


We devised an experiment to investigate how much the choice of synchronization barrier width affects our results, and also how much it affects simulation time.
In the experiment, we vary the value of the CSMB and compare average STP, HMS and MPKI values for all 4-core workloads.
Figure~\ref{fig:results:csmb} contains plots for both STP and HMS relative to the default 100 cycle barrier.
From the graph, it is apparent that lowering the value below 100 cycles causes negligible variations in our average results. 
The most noticeable is PIPP; that varies by about 0.2\% with a tighter barrier interval.
Increasing the interval to 1000 cycles results in a more noticeable difference in measurements.
For both HMS, shown in Figure~\ref{fig:results:csmb:hms}, and MPKI, not shown, the trends are the same.

The variance in simulation walltime when we vary CSMB values, as shown in Figure~\ref{fig:results:csmb:walltime}, is as expected. 
When lowering the barrier interval we measure an increase in average walltime.
Increasing the barrier value causes a slight decrease in walltime.
We observe that the performance gain by increasing the barrier is small compared to the result variation. 
When decreasing the barrier, the opposite is true; the result variation is small compared to the walltime increase.
These observations suggest that a barrier width of 100 cycles is a good trade-off between accuracy and walltime.