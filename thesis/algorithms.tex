
\chapter{Cache Management Algorithms}
\label{cpt:algorithms}


A cache management algorithm manages the storage space in a cache.
It decides where to store new data blocks and which of the existing blocks are evicted to make room for new blocks.
Some algorithms are thread-aware and geared towards shared caches.
Others are thread-agnostic and work both for shared and private caches.
Some have advanced optimization goals such as \gls{qos} while others use simpler metrics like miss minimization.
Algorithms proposed for shared caches may in general be divided into two groups, those that explicitly divide storage space between cores sharing the cache and those that do not.
The term cache replacement algorithm is often used to describe algorithms that do not divide the storage space while the term cache partitioning algorithm describe algorithms that do divide the space.
Throughout this paper, we will use the two terms interchangeably.

The field of cache management is well researched, and there exists a large number of proposed algorithms.
In this thesis, we present a few recently proposed algorithms and compare their performance.
We will also present \gls{lru}, an algorithm that is thread-agnostic and widely used both in private and shared caches today.
Table~\ref{tbl:algorithms} lists the selected algorithms.
Only algorithms that optimize for fewer cache misses are included.
This metric is easy to measure and also makes it easy to compare the various algorithms.
Also, we only consider algorithms that target conventional caches, as they are designed in \glspl{cmp} today.
This limitation makes the comparison of results from different algorithms easier.
Also, we avoid having to extend our simulator with a new cache type, which would be unfeasible given the time constraints of this thesis.

\begin{table}[thb]
\begin{tabular}{|p{1.4cm}|p{0.5cm}|p{0.8cm}|p{1.2cm}|p{1.2cm}|p{1.4cm}|p{1.2cm}|p{1.0cm}|}
\hline
\multicolumn{1}{|c|}{Name} & \multicolumn{1}{c|}{Year} & \multicolumn{1}{c|}{Thread} & \multicolumn{1}{c|}{Repl.} & \multicolumn{1}{c|}{Insert.} & \multicolumn{1}{c|}{Promo.} & \multicolumn{1}{c|}{Hardware}    & \multicolumn{1}{c|}{Partition}     \\
\multicolumn{1}{|c|}{}          & \multicolumn{1}{c|}{}          & \multicolumn{1}{c|}{aware}  & \multicolumn{1}{c|}{policy}      & \multicolumn{1}{c|}{policy}    & \multicolumn{1}{c|}{policy}    & \multicolumn{1}{c|}{overhead\footnotemark}  & \multicolumn{1}{c|}{}       \\ \hline
\gls{dip}                             & 2007                           & No                          & \gls{lru}                              & \gls{lip}/ \gls{bip}                        & Promote to \gls{mru}            & 1 counter, set dueling    & No            \\ \hline
\gls{tadip}                           & 2008                          & Yes                         & \gls{lru}                              & \gls{lip}/ \gls{bip}                        & Promote to \gls{mru}            & 1 counter per core, set dueling  & No          \\ \hline
\gls{drrip}                           & 2010                          & Yes                         & \gls{lru} approx.                      & \gls{srrip}/ \gls{brrip}                    & Stepwise promotion            & 1 counter per core, set dueling  & No          \\ \hline
\gls{nucache}                         & 2011                         & No                          & \gls{lru} + DeliWays     & \gls{lip}                            & Promote to \gls{mru}                 & NUTrack                            & No    \\ \hline
\gls{ucp}                             & 2006                           & Yes                         & Per core \gls{lru}                     & \gls{lip}                            & Promote to \gls{mru}                 & \gls{umon}, 1 \gls{atd} per core   & Yes                \\ \hline
\gls{pipp}                            & 2009                           & Yes                         & \gls{lru}                              & Utility position               & Stepwise promotion            & \gls{umon}, 1 \gls{atd} per core, random generator & Yes \\ \hline
\gls{prism}                           & 2012                           & Yes                         & Per core \gls{lru}                  & \gls{lip}                            & Promote to \gls{mru}                 & 1 \gls{atd} per core, random generator                       & Yes  \\ \hline
\gls{clu}                             & 2014                           & Yes                         & \gls{lru}                              & \gls{lip}/ \gls{bip}                        & Promote to \gls{mru}/ Skip            & \gls{umon} \~3 \glspl{atd} per core                & Yes  \\ \hline
\end{tabular}
\caption{Overview of Cache Management Algorithms.}
\label{tbl:algorithms}
\end{table}
\clearpage

It is possible to divide all algorithms included in this evaluation into three distinct policies:
\begin{itemize}
\item The replacement policy specifies which block a cache set evicts when inserting a new block into that set.
\item The insertion policy specifies the state of new blocks after insertion into the cache set.
\item The promotion policy specifies how the state of a block changes following an access from a processor core.
\end{itemize}
In the following sections, we will explain how each of the selected cache partitioning algorithms work, with an emphasis on this division to make comparisons easier.

\section{Cache Replacement Algorithms}
This section covers cache replacement algorithms, or algorithms that do not explicitly divide the available cache space between cores.


\footnotetext{Simplified hardware overhead compared to an \gls{lru} managed cached.}


\subsection{LRU}
\label{sec:background:algorithms:lru}


\begin{figure}[ht]
    \centering
    \begin{tabular}{|p{2cm}|p{2cm}|p{2cm}|p{2cm}|}
        \hline
        A & B & C & D \\
        \hline
    \end{tabular}
    \begin{tabular}{p{2cm}p{2cm}p{2cm}p{2cm}}
        MRU (3) & 2 & 1 & LRU (0) \\
            &  &  & \bf{Hit C}
    \end{tabular}    

    \begin{tabular}{|p{2cm}|p{2cm}|p{2cm}|p{2cm}|}
        \hline
        C & A & B & D \\
        \hline
    \end{tabular}
    \begin{tabular}{p{2cm}p{2cm}p{2cm}p{2cm}}
        MRU (3) & 2 & 1 & LRU (0) \\
            &  &  & \bf{Miss E}
    \end{tabular}    

    \begin{tabular}{|p{2cm}|p{2cm}|p{2cm}|p{2cm}|}
        \hline
        E & C & A & B \\
        \hline
    \end{tabular}
    \begin{tabular}{p{2cm}p{2cm}p{2cm}p{2cm}}
        MRU (3) & 2 & 1 & LRU (0) \\
            &  &  & \bf{Hit E}
    \end{tabular}    

    \begin{tabular}{|p{2cm}|p{2cm}|p{2cm}|p{2cm}|}
        \hline
        E & C & A & B \\
        \hline
    \end{tabular}
    \begin{tabular}{p{2cm}p{2cm}p{2cm}p{2cm}}
        MRU (3) & 2 & 1 & LRU (0) \\
    \end{tabular}    

    \caption{LRU managed 4-way cache set}
    \label{fig:background:lru_example}
\end{figure}

Least recently used (LRU) replacement, or some simplification of LRU, is one of the dominant cache replacement algorithms in hardware today. 
Normally when presenting a new cache partitioning algorithm, LRU is used as the baseline for comparions~\cite{Jaleel2010,Qureshi2006,Qureshi2007}.
The LRU algorithm relies on the temporal locality of data accesses; it assumes that recently accessed data has a higher reuse frequency then less recently accessed data.
When a cache miss occurs, and a new cache line is stored in a cache set, the LRU algorithm will evict the least recently used cache block.
Theoretically one can envision a cache set managed by LRU as a stack, where recently accessed cache blocks are near the top and less recently accessed blocks are near the bottom.
The bottom position of the stack is the LRU position, and the top position is the most recently used (MRU) position. 
In a hardware implementation, the blocks are not stored in a sorted fashion, but additional storage bits are used to keep track of LRU positions.

Figure~\ref{fig:background:lru_example} shows how a 4-way cache set managed by LRU replacement handles three requests. 
Initially, the set contains four blocks; A, B, C and D. 
A is in the MRU position while D is in the LRU position.
The first request is for block C; this is a hit, and that causes the block to move to the MRU position, pushing both block A and B one step closer to the LRU position.
The second request is made to block E; this block is not present in the cache.
The LRU algorithm evicts block D at the LRU position and then places block E at the MRU position.
Finally, another request is made to block E. Nothing changes since the block already is at the MRU position.

LRU is a simple replacement algorithm that is usable in both private and shared caches.
In shared caches, LRU will favor access frequency, giving cores that issue many cache requests more cache space then those how issue fewer requests. 
In some cases, this might be the an acceptable solution. 
However, several thread aware replacement algorithms claiming to outperform LRU exists. 
In this thesis, we will investigate several such algorithms and compare their performance against the simple LRU algorithm.

\todo{Mention Pseudo-LRU}


\subsection{Common Memmory Access Patterns}
In the following sections, we will present various cache partition algorithms that attempt to improve cache performance compared to the commonly used LRU algorithm.
In order to efficiently explain the strengths of the various partition algorithms, we will first define four different memory access patterns; recency-friendly, trashing, streaming and combined access patterns.

\paragraph{Recency-friendly}
As explained in Section~\ref{sec:background:algorithms:lru} the LRU algorithm assumes that recent memory accesses have a higher chance of reuse than less recent memory accesses.
We call this the recency-assumption, and we call memory access patterns that conform to this assumption recency-friendly access patterns.
Figure~\ref{fig:background:algorithms:rf_pattern} illustrates an example memory access pattern that is said to be recency-friendly.
Assuming LRU replacement, the cache will provide hits for all accesses every time the pattern repeats.

\begin{figure}[ht]
\centering
\begin{equation}
p_{rf} = (a_0 a_1 ... a_{k-1})^N
\end{equation}
\caption{Recency-friendly access pattern to a single cache set (k $<=$ number of ways, N $>$ 1)}
\label{fig:background:algorithms:rf_pattern}
\end{figure}

\paragraph{Trashing}
A trashing memory pattern is one that repeats in a similar manner to a recency-friendly pattern, but with more blocks per cycle than the number of cache ways. 
Figure~\ref{fig:background:algorithms:tr_pattern} shows an example of one such pattern, the only thing separating this from the previous pattern in the value of k.
In an LRU managed cache, an access pattern similar to this will never hit because all blocks are evicted before they are re-referenced.
An optimal replacement algorithm might keep some of the working set in the cache, providing hits for parts of the access pattern.
\begin{figure}[ht]
\centering
\begin{equation}
p_{rf} = (a_0 a_1 ... a_{k-1})^N
\end{equation}
\caption{Trashing memory access pattern to a single cache set (k $>$ number of ways, N $>$ 1)}
\label{fig:background:algorithms:tr_pattern}
\end{figure}

\paragraph{Streaming}
A streaming memory access pattern is one that has not re-references, or where the period is so large that no pattern is detectable.
Figure~\ref{fig:background:algorithms:st_pattern} illustrates one such pattern, where k is infinite.
Neither LRU or any other caching algorithm can provide hits for such an access pattern, simply because there are no re-references.

\begin{figure}[ht]
\centering
\begin{equation}
p_{rf} = (a_0 a_1 ... a_{k-1})
\end{equation}
\caption{Streaming memory access  pattern (k = $\infty$) }
\label{fig:background:algorithms:st_pattern}
\end{figure}

\paragraph{Combined}
In reality, memory access patterns can more complex than the simple examples shown above.
A single program might, for instance, behave both in a recency-friendly and streaming fashion. 
An example would be a program performing a reduction over a large dataset.
Most accesses would be streaming as the program iterates over the dataset, but some accesses will exhibit recency-friendly behavior like when the program stores temporary results to memory during the iteration.

For shared caches the observed pattern is the union of accesses from all cores, this can result in even more complex patterns.
A particular non-optimal situation for an LRU managed shared cache is when multiple cores execute recency-friendly applications, and one single core execute a streaming application.
In this case, the streaming application will constantly clear the cache, degrading performance of the recency-friendly applications.
An algorithm that could detect the one streaming application and handle it as a special case could potentially increase performance of the recency-friendly applications without affecting the performance of the streaming application.
Some of the algorithms we cover in this report will attempt to detect streaming applications.
\section{DIP}
\label{sec:algorithms:dip}

Dynamic Insertion Policy (DIP)~\cite{Qureshi2007} was originally proposed by M. K. Qureshi in 2007
The DIP algorithm views the cache set as a stack, as in LRU.
Replacement and promotion policies are equal to LRU, DIP evicts the block at the LRU position, and following a cache hit a block moves to the MRU position.
In contrast to LRU, DIP is a combination of two insertion policies, the standard LRU insertion policy (LIP) and Binominal Insertion Policy (BIP).
LIP inserts new blocks at the MRU position.
BIP inserts new blocks either at the LRU position or with a small probability, $p = \frac{1}{32}$, at the MRU position. 
The overall DIP algorithm switches between the two insertion policies by always using the one that is expected to cause fewer cache misses.

By mostly inserting at the LRU position the BIP insertion policy can theoretically handle trashing memory access patterns.
BIP inserts most of the new blocks in the LRU position, and the upper part of the LRU stack can contain blocks that have been re-referenced.
In a trashing access pattern, this results in part of the working set residing in the upper part of the stack while rest is inserted at the LRU position and evicted at the next miss.
By sometimes inserting at the MRU position BIP will give blocks not referenced by the next miss a chance to stay in the cache. 
This will also force stale cache blocks in the upper part of the stack to move towards the LRU position.

The authors of DIP present several methods to detect the best replacement algorithm, one of them is set-dueling.
Set-dueling is implemented by having some sets of the cache always use either BIP or LRU. 
A counter tracks the performance of the dueling sets.
Misses in LRU sets will increment the counter and misses in BIP sets will decrement the counter.
The MSB of the counter can then be used to select the optimal algorithm.
If the MSB is one, an overweight of misses in LRU sets are occurring, and BIP is the optimal algorithm. 
If the MSB is zero, then an overweight of BIP misses are occurring, and LRU is the optimal algorithm.
Another solution is to utilize two Auxilliary Tag Directories (ATDs) which is a structure equal to the Tag directory of the cache itself.
The two ATDs run one algorithm each, and all cache accesses are also executed on the ATDs.
Again a counter controlled by misses in either ATD is used to select the optimal algorithm for the main cache.
The main advantage of using a ATD is that all available information is used when selecting between BIP and LIP.
Also the entire cache will always use the best algorithm, where in set-dueling a faction of the sets will always run the non-optimal algorithm.
However, ATDs required additional storage space, where set-dueling uses the existing store space and only requires counters.
\subsection{TADIP}
\label{sec:background:algorithms:tadip}

Thread Aware Dynamic Insertion Policy (TADIP)~\cite{Jaleel2008} proposed by A. Jaleel et al. in 2008 is an extension of DIP~\cite{Qureshi2007} originally proposed by M. K. Qureshi in 2007.
In the following paragraphs, we will first present DIP and then present the additions needed to make DIP perform well for shared caches.

\paragraph{DIP}

The DIP algorithm views the cache set as a stack, as in LRU.
On a cache miss, DIP evicts the block at the LRU position. 
In contrast to LRU, DIP is a combination of two insertion policies, the standard LRU insertion policy (LIP) and Binominal Insertion Policy (BIP).
LIP inserts new blocks at the MRU position.
BIP inserts new blocks either at the LRU position or with a small probability, $p = \frac{1}{32}$, at the MRU position. The overall DIP algorithm switches between the two insertion policies by always using the one that is expected to cause fewer cache misses.

By mostly inserting at the LRU position the BIP insertion policy can theoretically handle trashing memory access patterns.
BIP inserts most of the new blocks in the LRU position, and the upper part of the LRU stack is reserved for blocks that have been re-referenced.
In a trashing access pattern, this results in part of the working set residing in the upper part of the stack while rest is inserted at the LRU position and evicted at the next miss.
By sometimes inserting at the MRU position BIP will give blocks not referenced by the next miss a chance to stay in the cache. 
This will also force stale cache blocks in the upper part of the stack to move towards the LRU position.

The authors of DIP present several methods to detect the best replacement algorithm. 
One such technique is called set-dueling.
Set-dueling is implemented by having some sets of the cache use either BIP or LRU. 
A counter tracks the performance of the dueling sets.
Misses in LRU sets will increment the counter and misses in BIP sets will decrement the counter.
The MSB of the counter can then be used to select the optimal algorithm.
If the MSB is one, an overweight of misses in LRU sets are occurring and BIP the optimal algorithm. 
If the MSB is zero, then an overweight of BIP misses are occurring, and LRU is the optimal choice.
Another solution is to utilize two Auxilliary Tag Directories (ATDs) which is a structure equal to the Tag directory of the cache itself.
The two ATDs run one algorithm each, and all cache accesses are also executed on the ATDs.
Again a counter controlled by misses in either ATD is used to select the optimal algorithm for the main cache.

\paragraph{TADIP}

One issue with DIP is that it does not consider from which core a cache accesses originate.
In a workload with multiple benchmarks, some might be recency-friendly while others are not. 
If a shared cache is managed by DIP, then the algorithm choice is made based on the sum of the cache accesses and then applied equally to all benchmarks.
The authors of TADIP recognized that improvements in performance could be achieved by selecting DIP algorithm on a per core basis when utilized on a shared cache.

When selecting the best performing algorithm per core, the ATD technique requires two ATDs per core sharing the cache. 
This quickly becomes too expensive to be practical.
Set-dueling in DIP requires a minimum of two sets, one running LRU (1) and one running BIP (0). 
With two cores, the number of combinations rises to four (00, 01, 10, 11).
When the number of cores continues to increase this also seems to be an impractical solution.
As a result, the authors of TADIP suggests two selection techniques that reduce the number of sets required. 
Both solutions have one counter per core sharing the cache.
This counter is used to select the optimal policy for that core.

TADIP-Isolated (TADIP-I) has one set per core running BIP for that core and LRU for all others.
In addition to these N sets, a single set runs LRU for all cores. 
A miss in the LRU set will increment all the core counters while a miss in the core specific set will decrement the counter for the specific core.
For a large N, this solution requires significantly less duel-sets compared to having one per combination. 
Because it assumes that all other cores run LRU, the effect different cores with different policies might have on each other is ignored.

TADIP-Feeback (TADIP-F) attempts to reduce the error caused by the assumption of other cores by having two sets per core, a total of 2N.
Per core, one set runs LRU the other runs BIP, any inserts from other cores uses the currently best algorithm for that core.
A miss in the LRU set for a core will increment that cores counter, a miss the BIP set will decrement the counter.

\todo{Create an illustration like the one in the TADIP paper!}

Figure~\todo{ref} shows an illustration of a cache managed by DIP, TADIP-I, and TADIP-F. 
As is shown the duel-sets effect a set of counters, these counters give the optimal policy, P, per core. 
The remaining sets use the set of current optimal replacement policies.
From this point on when we refer to TADIP in this report we assume TADIP-F unless otherwise stated.

When detailing their implementation, the authors of TADIP show a simple hash function used to select dueling sets.
On their 4096 set cache, they use the formulas shown below.
Here set index is a number from 0-4095, core\_id is the zero-indexed id of the requester core and cores is the total number of cores sharing the cache.
If BIP or LIP is true, then the set is a duel set for the given core, and the policy forced to either BIP or LIP.
If both BIP and LIP is false, then the set is a normal follower set and utilized the current best algorithm for the given core.
From the algorithm, it is clear that the original paper use a total of 32 groups of duel sets spread evenly throughout the cache.
\begin{figure*}[ht]
\begin{equation}
LIP = set\_no[11:7] + core\_id == set\_no[6:0]
\end{equation}
\begin{equation}
BIP = set\_no[11:7] + core\_id + cores == set\_no[6:0]
\end{equation}
\begin{equation}
FOLLOWER = !LIP + !BIP
\end{equation}
\end{figure*}

\todo{We should run an experiment finding the best binomial parameter for our workloads, also we need to find the optimal number of duel sets. The original paper chooses 1/32, compare this to our result. How is this experiment best executed? Does using only 4-benchmark workloads approximate or should we also use 8- and 16-benchmakr workloads? If so we have to create them.}
\section{DRRIP}
\label{sec:background:algorithms:drrip}

\begin{figure}[ht]
    \centering
    \begin{tabular}{|p{2cm}|p{2cm}|p{2cm}|p{2cm}|}
        \hline
        A & B & C & D \\
        \hline
    \end{tabular}
    \begin{tabular}{p{2cm}p{2cm}p{2cm}p{2cm}p{2cm}p{2cm}}
        & 2 & 1 & 1 & 0 & \bf{Hit C} \\
        &   &   &   &   &
    \end{tabular}    

    \begin{tabular}{|p{2cm}|p{2cm}|p{2cm}|p{2cm}|}
        \hline
        A & B & C & D \\
        \hline
    \end{tabular}
    \begin{tabular}{p{2cm}p{2cm}p{2cm}p{2cm}p{2cm}p{2cm}}
        & 2 & 1 & 0 & 0 & \bf{Miss E} \\
        &   &   &   &   &
    \end{tabular}     

    \begin{tabular}{|p{2cm}|p{2cm}|p{2cm}|p{2cm}|}
        \hline
        E & C & A & B \\
        \hline
    \end{tabular}
    \begin{tabular}{p{2cm}p{2cm}p{2cm}p{2cm}p{2cm}p{2cm}}
        & 2 & 2 & 1 & 1 & \bf{Hit E} \\
        &   &   &   &   &
    \end{tabular}    

    \begin{tabular}{|p{2cm}|p{2cm}|p{2cm}|p{2cm}|}
        \hline
        E & C & A & B \\
        \hline
    \end{tabular}
    \begin{tabular}{p{2cm}p{2cm}p{2cm}p{2cm}p{2cm}p{2cm}}
        & 1 & 2 & 1 & 1 & 
    \end{tabular} 

    \caption{DRRIP managed 4-way cache set (M=2, static insertion)}
    \label{fig:background:drrip_example}
\end{figure}

Dynamic Re-Reference Interval Prediction (DRRIP) was first proposed by A. Jaleel et al.~\cite{Jaleel2010} in 2010.
DRRIP does not utilize the concept of an LRU stack as both LRU and TADIP does.
In DRRIP, each cache block has a number associated with it, called re-reference interval.
The re-reference interval is a measure of the time interval between now and the next time the algorithm expects a block to be re-referenced.
It is represented as a value between 0 and $2^M - 1$ where M is a configurable parameter.
A value of 0 indicates a near re-reference interval, the algorithm expects the block to be re-referenced in the near future.
The value $2^M - 1$ indicates a distant re-reference interval while the value of $2^M - 2$ indicates a long re-reference interval.
Multiple blocks may have the same re-reference interval. 
Hence, blocks are not strictly ordered as in the LRU stack.
By setting $M=1$ DRRIP degrades into the Not Recently Used (NRU)~\cite{Microsystems2007} algorithm, which is used on the UltraSPARC T2.

The replacement policy of DRRIP is to scan all blocks and evict the first one found with a distance re-reference interval.
If no blocks have a distant re-reference interval the re-reference interval of all blocks are incremented by one and the scan starts over.
This process repeats until the algorithm finds a victim block.
If multiple blocks are potential victims, the algorithm uses the scan order as a tie-breaker.
In the original paper, the authors specify that the leftmost potential block, the one with a lower block index, is the victim.

DRRIP's promotion policy is to decrement the re-reference interval of the accessed block.
By doing this DRRIP utilized access frequency rather than access time to calculate the re-reference interval.
Hence, in order to reach a near re-reference interval a block has to have a high access frequency.
This is in contrast to LRU, where a block will move to the MRU position following a hit, independent of the previous access history.

The insertion policy of DRRIP, like TADIP, is composed of two different policies and a selection mechanism.
Static RRIP (SRRIP) will always insert new block with a long re-reference interval. 
Depending on the state of the cache blocks inserted by SRRIP will therefore potentially have other blocks behind it with a lower re-reference interval.
Binominal RRIP (BRRIP) is analog to BIP under TADIP.
BRRIP with either insert new blocks with a distant re-reference interval or, with a small probability, insert like SRRIP with a long re-reference interval.
Like BIP, BRRIP will allow trashing access patterns to keep some of the working set in the cache and hence improve performance over SRRIP.
Selecting between the two insertion policies can be done in multiple ways, like covered in the description of TADIP.
The authors use set-dueling in their original paper, and we opt to do this in our implementation as well.
Since there is lacking information regarding selection of duel sets in the original paper, we utilize the hashing function that we presented for TADIP in our implementation.

Figure~\ref{fig:background:drrip_example} shows an example cache managed by DRRIP.
In the example, M is set to 2, making the distant re-reference interval 3 and the long re-reference interval 2. 
In addition, we assume static insertion thought the example.
Initially, there are four blocks A, B, C and D with re-reference intervals 2, 1, 1 and 0.
First an access hits the C block, and its value decrements to 0.
Next a miss to block E occurs, and as no block has a re-reference interval of 3 the value of all blocks is incremented by one. 
A then has a value of 3 and is evicted, E in inserted in its place with a value of 2.
Finally, a hit to E occurs causing its value to decrease.

\todo{Add more indepth coverage of the math behind DRRIP and in which situtations it can acheive stream and trash resistance.}
\section{NUCache}
\label{sec:algorithms:nucache}

\section{Cache Partitioning Algorithms}
This section covers cache partitioning algorithms.
In contrast to the replacement algorithms, these algorithms explicitly assign a set number of blocks in each cache set to each core.

\section{UCP}
\label{sec:algorithms:ucp}

Utility Cache Partition~\cite{Qureshi2006} (UCP) was first presented by M. Qureshi and Y. Patt in 2006. 
UCP uses the concept of utility when assigning ways to a core.
Using a utility monitor (UMON), UCP calculates the number of ways assigned to each core.
UCP then uses the same insertion and promotion policy as LRU.
The replacement policy is as in LRU with two modifications:
First if the number of blocks owned by the requesting core is less than the number of ways assigned to it, then the least recently used block that is not assigned to the requester core is replaced.
If however the number of blocks owned is greater than or equal to the number of assigned ways the replacement algorithm selects the  least recently used block of those owned by the requester.
As a result, the division between cores in each set move towards the global allocation on cache misses, allowing cores to use more than allocated if other cores do not utilize their allocated space.

The UMON is the core of the UCP algorithm.
It consists of one ATD per core sharing the cache. 
The ATD is managed by normal LRU replacement and has one access counter per way.
Whenever a cache request hits in the ATD the access counter index by the LRU stack position of that block is incremented.
Hence, on a access to the MRU block the first counter is incremented.
In addition to the ATDs, there is a monitor circuit that has access to the counters and using them calculates how to partition the cache at set intervals. 
In the original paper, the authors recalculate the partitioning every 5M cycles.

The original paper proposes several algorithms for determining optimal partitioning based on the counter data. 
Algorithm~\ref{alg:algorithms:ucp} is the Lookahead Algorithm they propose.
This algorithm is the one the original authors propose for situations with more than two cores sharing a cache.
The algorithm assigns cache ways based on an increase in marginal utility.
While there are more ways to distribute, the algorithm calculates the maximum marginal utility achievable by each core. 
The core with the highest value wins and is assigned as many ways as needed to achieve the increase.
The algorithm continue until all ways have been assigned.
Lines 27-28 calculate the actual marginal utility. 
First the number of misses prevented by increasing the allocation from a to be is found.
With the counters available the algorithm can simply sum the values of counters a to $b-1$, as the number of hits in the new sets must equal the number of misses prevented.
To find the marginal utility, this value is then divided by the number of sets introduced.
The rest of the algorithm is simply a greedy algorithm selecting the best core each iteration.
After a reallocation of cache ways, the ATD counters are all halved.
By doing this, the UMON will keep historical data for future decisions, while prioritizing data from the current period.
While this algorithm could choose to assign 0 ways to one or more cores both the authors' implementation~\cite{Qureshi2006} and our implementation forces at least one way per core.
It is unclear from their paper why the original authors did this, but in our case it is done because our simulator implements inclusive caches.

\begin{algorithm}[ht]
\caption{UMON Lookahead Algorithm}
\label{alg:algorithms:ucp}
\begin{algorithmic}[1]
\State $balance\gets N$ /* Number of ways */
\State $allocations[i]\gets 0$  /* for each core $i$ */
\While {$balance$}
    \ForAll {$cores\ i$}
        \State $alloc\gets allocatations[i]$
        \State $max\_mu[i]\gets \Call{get\_max\_mu}{i, alloc, balance}$
        \State $blocks\_req[i]\gets$ min blocks to get max\_mu[i] for i
    \EndFor
    \State $winner\gets$ application with the maximum value of max\_mu
    \State $allocations[winner] += blocks\_req[winner]$
    \State $balance -= blocks\_req[winner]$
\EndWhile
\State \Return alloactions
\State

\Function{get\_max\_mu}{$i, alloc, balance$}
    \State $max\_mu\gets 0$
    \For{ii = 1; ii <= balance; ii++}
        \State $mu\gets \Call{get\_mu\_value}{p, alloc, alloc+ii}$
        \If{$mu \ge max\_mu$}
            \State $max\_mu\gets mu$
        \EndIf
    \EndFor
    \State \Return{$max\_mu$}
\EndFunction
\State

\Function{get\_mu\_value}{$p, a, b$}
    \State $U\gets$ change in misses for application p when number of blocks assigned to it increases from a to b
    \State \Return{$\frac{U}{b-a}$}
\EndFunction
\end{algorithmic}
\end{algorithm}


\subsection{PIPP}
\label{sec:background:algorithms:pipp}

Promotion/Insertion Pseduo-Partitioning~\cite{Xie2009} (PIPP) proposed by Y. Xie and G. Loh in 2009 is a algorithm based on a slightly modified UMON circuit and a novel insertion and promotion policy.
The UMON changes are to enable stream detection.
Where the UCP algorithm only handles streaming applications indirectly, by assigning few ways because of a low hit rate in the ATDs, PIPP's UMON actively detects streaming applications.
Stream detection is implemented by adding an additional counter that counts the total number of cache misses in the ATD.
An application is then deemed to be streaming if either the number of misses over during a single allocation period or if the miss rate over the same limit is higher than a set maximum.

PIPP like UCP views the cache set as a LRU stack.
The replacement policy is as in LRU, but the insertion and promotion policy is novel.
The insertion policy inserts new blocks $\pi_n$ blocks from the LRU position. 
Here $\pi_n$ is the number of ways assigned to the nth core.
In a 4-way cache dual-core setup where both cores are assigned two ways, PIPP will insert all new blocks from either core in the second to last position in the stack. 
In this situation, the two top positions in the cache stack can only be reached by a cache block through promotion.
Block promotion in PIPP works similar promotion in DRRIP. 
On a access, a block has a chance, $p_{prom} = \frac{1}{32}$, to move one position upwards in the stack unless it is already at the MRU position.

Where the insertion policy of UCP ensures that a core cannot claim more cache blocks than it needs, the PIPP policies do not enforce this.
However, cores with more ways assigned to it will insert its blocks higher up in the stack. 
The core with the highest number of ways assigned will not have any insertion competition pushing its blocks out of the cache.
The only way blocks from this core can be pushed out is by other blocks from the same core, or by blocks from other cores that are re-referenced repeatably.
While two cores with the same allocations will both have an equal chance of keeping their blocks in the cache, as they both insert at the same position.
Statistically a core with a lower allocation, inserting at a lower position in the stack, should also on average own fewer blocks in the cache compared to a core with a higher allocation.
However, the access frequency of cores can cause a core with a low allocation to own most or all blocks in some cache sets if the other cores access this set at a much slower rate.
This way PIPP obtains what the original authors call pseudo partitioning, where overall a higher allocation will statistically result in more cache space.

When the UMON detects an core that is streaming PIPP will no longer insert blocks from this core at the position given by the allocation.
A special insertion position, $\pi_{stream}$, is used for all streaming cores.
$\pi_{stream}$ is set to the number of cores currently streaming. 
By inserting at this fixed position, PIPP attempts to limit the interference the streaming core has on the other cores.
Blocks from streaming applications have a reduced chance of promotion after an access, $p_{stream} = \frac{1}{128}$.
In the special case where all cores are streaming, and there are no cores to protect, PIPP uses the value 0 for $\pi_{stream}$.

\todo{Example}
\subsection{PriSM}
\label{sec:algorithms:prism}

\gls{prism}~\cite{Manikantan2012} was first presented in 2012.
\gls{prism} is a framework for cache management with optimization algorithms targetting multiple performance goals.
The original paper presents hit maximization, fairness and \gls{qos} goals.
We will focus on the hit maximization algorithm , or miss minimization algorithm, as all other algorithms in this thesis also targets this goal.
\gls{prism} utilizes \glspl{atd} to estimate private cache performance for each of the cores.
The \gls{atd} will keep track of total misses and hits.
It will not track hits per cache way like the \glspl{atd} in \gls{ucp} and \gls{pipp}.
In addition to the \glspl{atd}, the algorithm requires three counters per core tracking hits, misses and number of blocks owned by the core in the actual cache.
\gls{prism} utilizes the same insertion and promotion policies as \gls{lru}, but the replacement policy is optimized based on the \gls{atd} and the optimization target.

The replacement algorithm of \gls{prism} utilizes eviction probabilities, $E_i$ ($\sum{E_i} = 1$), assigned to each core when selecting a victim block.
On replacement, a victim core is first selected by a random draw using the eviction probabilities.
The \gls{lru} block owned by the victim core within the cache set is the eviction target.
In the case where the selected target does not own a block in the set, all blocks owned by cores with $E_i > 0$ are considered, and the \gls{lru} of these is the eviction target.
At set intervals, an optimization algorithm determines the eviction probability, $E_i$, for each core.
The original paper recalculated $E_i$ values at every 10000 cache miss.

The insertion and promotion policy of \gls{prism} is equal to \gls{lru}.
On insertion, a block is promoted to the \gls{mru} position, and on any subsequent accesses the block is again promoted to \gls{mru} unless it already has that position.

Selecting an eviction probability $E_i$ for each core is done by considering how the eviction probability will effect a core's usage of the cache.
Consider an interval of W misses where each core contributes a fraction of the misses, $M_i$.
At the start of the interval the blocks owned by $core_i$ equals a fraction $C_i$ of the total number of blocks in the cache.
If we do not evict any blocks owned by $core_i$ during the interval, then at the end of the interval the core owns a fraction $T_i$ of the cache.
$T_i$ is known as the target allocation, and is expressed by $T_i = C_i + M_i * W/N$. 
Here $M_i * W$ is the number of misses caused by $core_i$ during the interval, which also is the number of blocks inserted by the core.
$N$ is the total number of blocks in the cache, and the fraction $M_i * W/N$ equals the fraction of the cache claimed by $core_i$ during the interval.
If the core has a non-zero eviction probability, then this formula extends into $T_i = C_i + (M_i - E_i) * W/N$.
As noted, \gls{prism} defines three optimization targets, each one of these is responsible for calculating the optimal $T_i$ that will fulfill the optimization target.
Rearranging the above formula for $E_i$ yields: $E_i = (C_i - T_i) * N/W + M_i$.
Algorithm~\ref{alg:algorithms:prism} shows how $T_i$ values are calculated for hit maximization.
It is a relatively simple algorithm that will adjust the target occupancy based on the current occupancy and the potential for gaining more hits.

\begin{algorithm}[ht]
\caption{PriSM Hit Maximization.}
\label{alg:algorithms:prism}
\begin{algorithmic}[1]
\State $N$ /* Number of cores */
\ForAll{$cores\ i$}
    \State $PotentialGain[i]\gets StandAloneHits[i] - SharedHits[i]$
\EndFor
\State $TotalGain\gets \sum{PotentialGain}$

\ForAll{$cores\ i$}
    \State $T_i\gets C_i * (1 + \frac{PotentialGain[i]}{TotalGain})$
\EndFor
\State $T_i = \frac{T_i}{\sum{T}}$ /* Normalize target occupancy */
\end{algorithmic}
\end{algorithm}

While we have presented \gls{prism} based on \gls{lru} replacement, as done in the original paper, it should be noted that \gls{prism} is not dependent on this underlying replacement algorithm.
Any algorithm is usable, as long as it is augmented to prioritize the selected victim during replacement.
The algorithm run on the \glspl{atd} has to be the same as the underlying algorithm in the \gls{prism} implementation.

\subsection{CLU}
\label{sec:algorithms:clu}

CLU, short for Co-Optimizing Locality and Utility, is an algorithm presented by D. Zhan et al.~\cite{Zhan2014} in 2014.
The authors of CLU recognize that recent research in LLC partitioning has followed two distinct directions.
Some publications optimize for access locality and attempts to improve performance by changing the lifetime of blocks in LRU managed caches.
DIP, TADIP, and NUCache are three such solutions that use novel methods to reduce or extend the lifetime of blocks in an elsewise LRU managed cache.
Other publications recognize the usefulness of utility and do way-partitioning between cores based on their utility values.
Examples here are UCP and PIPP.
Because the stack property of LRU is the base of the utility calculations in both PIPP and UCP~\cite{Qureshi2006, Xie2009}, they cannot evaluate the utility of algorithms such as DIP's BIP.

The authors of CLU presents a novel approach for calculating the utility curve of a BIP managed cache.
BIP, as covered earlier, is one of the two insertion policies under DIP and TADIP. 
BIP violates the stack property of LRU by mostly inserting new rows at the MRU position, or at a low probability in LRU position.
In order to correctly measure the utility curve of a BIP managed k-way cache, one needs k ATDs; ATD($1$), ATD($2$), ... ATD($k$). 
Where ATD($x$) is an x-way ATD.
In contrast, the utility curve of an LRU managed cache can be found using one ATD, due to the stack property.
Having k ATDs per core sharing the LLC is not a realistic goal due to the required overhead.
The authors of CLU proposes a simplification where there are $m = log_2 k$ ATDs; ATD($1$), ATD($2^1$), ..., ATD($2^m$).
A linear increase between the sample points is assumed when calculating the final utility curve.
It should be noted that the storage overhead of m ATDs in total is less the twice the overhead of the single ATD($k$) required to sample the LRU curve.

CLU uses the two curves first to allocate ways to each core using the same algorithm as shown for UCP in section~\ref{sec:algorithms:ucp}.
The only difference is that the algorithm uses either the LRU of BIP value when estimating utility given an allocation, depending on which algorithm performs best.
During runtime, CLU works like UCP.
The only exception is that the cores ways are managed by either LRU or BIP, depending on which algorithm has the best utility value for the number of ways currently assigned to that core.

\todo{Create a sample plot with a dummy lru and bip utility line as an illustration}