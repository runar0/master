
\section*{Abstract}

The performance gap between processors and main memory has been growing over the last decades.
Fast memory structures know as caches were introduced to mitigate some of the effects of this gap.
After processor producers reached the limits of single core processors performance in the early 2000s, multicore processors have become common.
Multicore processors commonly share cache space between cores, and algorithms that manage access to shared cache structures have become an important research topic.
Many researchers have presented algorithms that are supposed to improve the performance of multicore processors by modifying cache policies.
In this thesis, we present and evaluate several recent and important works in the cache management field.
We present a simulation framework for evaluation of various cache management algorithms, based on the Sniper simulation system.
Several of the recently suggested algorithms are implemented and evaluated against the commonly used \gls{lru} replacement policy.
Our results suggest that \gls{lru} replacement in many cases is sufficient, and few of the evaluated algorithms consistently outperforms \gls{lru}.
We also show that \gls{ucp}, the oldest evaluated alternative to \gls{lru}, is the best performer for all our workloads.

\glsresetall
\clearpage

\section*{Sammendrag}
Ytelsesforskjellen mellom prosessorer og hovedminne har økt gjennom de siste tiår.
Raske minnestrukturer kjent som hurtigbuffer vart introdusert for å redusere effekten av den økende forskjellen.
Etter at produsenter møtte grensen for enkjerneprosessorytelse på starten av 2000-tallet har flerkjerneprosessorer blitt vanlige.
Flerkjerneprosessorer deler vanligvis noe hurtigbuffer mellom kjernene, og algoritmer som kontrolerer kjernenes tilgang til det delte området har blitt et viktig forskningsområdet.
Flere forskere har presenter algoritmer som skal kunne øke ytelsen til flerkjerneprosessorer ved å endre algoritmen som styrer hurtigbufferet.
I denne avhandlingen presenterer og evaluerer vi flere nylig publiserte og viktige arbeid som omhandler kontroll av delt hurtigbuffer.
Vi presenterer et simuleringssystem som vi bruker til å evaluerer flere algorithmer, basert på simuleringssystemet Sniper.
Flere av de nylige presenterte algoritmene er implementer og evaluert mot den ofte brukte algoritmen \gls{lru}.
Våre resultat viser til at \gls{lru} i flere tilfeller er tilstrekkelig, og at få av de implementerte algoritmene kan konsekvent produsere bedre resultat enn \gls{lru}.
Vi viser også at \gls{ucp}, den eldste av de implementerte alternativet til \gls{lru}, gir best resultat på alle våre testprogrammer.

\clearpage
